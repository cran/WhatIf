\documentclass[oneside,letterpaper,titlepage]{article}
\usepackage{makeidx}
\usepackage{graphicx}
\usepackage{natbib} 
\usepackage[reqno]{amsmath}
\usepackage{amssymb}
\usepackage{verbatim}
\usepackage{epsf}
\usepackage{url}
\usepackage{html}
\usepackage{dcolumn}
\usepackage{longtable}
\usepackage{vmargin}
\topmargin=0in

%\setpapersize{USletter}
\newcolumntype{.}{D{.}{.}{-1}}
\newcolumntype{d}[1]{D{.}{.}{#1}}
%\pagestyle{myheadings}
\htmladdtonavigation{
  \htmladdnormallink{%
    \htmladdimg{http://gking.harvard.edu/pics/home.gif}}
  {http://gking.harvard.edu/}}
\newcommand{\hlink}{\htmladdnormallink}

\bodytext{ BACKGROUND="http://gking.harvard.edu/pics/temple.jpg"}
\setcounter{tocdepth}{3}

\newcommand{\WhatIf}{\textsc{WhatIf}}

\title{\WhatIf: Software for Evaluating Counterfactuals\thanks{Available from
    http://GKing.Harvard.Edu/whatif.}}

\author{Heather Stoll\thanks{Assistant Professor of Political Science,
    University of California, Santa Barbara (3713 Ellison Hall,
    University of California, Santa Barbara, CA 93106;
    \texttt{http://www.polsci.ucsb.edu/faculty/hstoll/},
    \texttt{hstoll@polsci.ucsb.edu}).}  \and %
  Gary King\thanks{David Florence Professor of Government, Harvard
    University (Institute for Quantitative Social Science, 34 Kirkland
    Street, Harvard University, Cambridge MA 02138;
    \texttt{http://GKing.Harvard.Edu}, \texttt{King@Harvard.Edu},
    (617) 495-2027).}  \and %
  Langche Zeng\thanks{Professor of Political Science, University of
    California-San Diego (Department of Political Science, University
    of California, San Diego, 9500 Gilman Drive, La Jolla, CA
    92093-0521, \texttt{zeng@ucsd.edu}).}}

% rbuild: replace 'Version ' '\\' Version
\date{Version 1.5-4\\ \today}

%
\makeindex

\begin{document}\maketitle

\begin{rawhtml} 
  <p> [Also available is a down-loadable <a
  href="/whatif/docs/whatif.pdf">PDF</a> version of this entire
  document]
\end{rawhtml}

\tableofcontents
\clearpage

\section{Introduction}

\WhatIf\ implements the methods for evaluating counterfactuals
introduced in \citet{KinZen06} and \citet{KinZen07}:
\begin{quote}
  Gary King and Langche Zeng.  2006. ``\hlink{The Dangers of Extreme
    Counterfactuals}{http://GKing.Harvard.Edu/files/abs/counterft-abs.shtml},'' 
    \emph{Political
    Analysis} 14 (2):  131--159.
\end{quote}
and
\begin{quote}
  Gary King and Langche Zeng.  2007.  ``\hlink{When Can History Be
    Our Guide?  The Pitfalls of Counterfactual
    Inference}{http://GKing.Harvard.Edu/files/abs/counterf-abs.shtml},'' \emph{International
    Studies Quarterly} 51 (March):  183--210.
\end{quote}
The two papers overlap, with the first containing all the proofs and
technical material and the second having more pedagogical material and
examples.

Inferences about counterfactuals are essential for prediction,
answering ``what if'' questions, and estimating causal effects.
However, when the counterfactuals posed are too far from the data at
hand, conclusions drawn from well-specified statistical analyses
become based largely on speculation hidden in convenient modeling
assumptions that few would be willing to defend.  Unfortunately,
standard statistical approaches assume the veracity of the model
rather than revealing the degree of model dependence, which makes this
problem hard to detect.  

\WhatIf\ offers easy-to-apply methods to evaluate counterfactuals that
do not require sensitivity testing over specified classes of models.
If an analysis fails the tests offered here, then we know that
substantive inferences will be sensitive to at least some modeling
choices that are not based on empirical evidence, no matter what
method of inference one chooses to use.  Specifically, \WhatIf\ will
indicate whether a given counterfactual is an extrapolation (and
therefore risking more model dependence) or a (safer) interpolation.
Using an algorithm developed in \citet{KinZen06} to identify whether
counterfactual points are within the convex hull of the observed data,
this is feasible even for large numbers of explanatory variables.  It
will also compute either the Gower or Euclidian distances from the
counterfactuals to each observed data point.  The convex hull test can
additionally be used to approximate the common support of the
treatment and control groups in causal inference.  Numerical and
graphic summaries are offered.

\WhatIf\ has been incorporated in \hlink{MatchIt}{/matchit}, and also
works easily with \hlink{Zelig}{/zelig} output
\citep{HoImaKin07,ImaKinLau06,ImaKinLau07}.

\section{Installation}

\WhatIf\ requires \texttt{R} version 2.3.1 or later, available from
CRAN (\url{http://cran.r-project.org/}), and the package
\texttt{lpSolve}, also available from CRAN.  Installation of \WhatIf\ 
differs slightly by operating system.

\subsection{Windows}

Begin the installation process by launching \texttt{R}.  To install
the package \WhatIf\ as well as the package that it depends upon,
\texttt{lpSolve}, type:
\begin{verbatim}
  > install.packages("WhatIf", dependencies = TRUE)
\end{verbatim}
at the \texttt{R} command prompt.  This command installs the packages
from the CRAN respository set as part of your \texttt{R} options.  You
can see what its current value is by calling
\begin{verbatim}
  > getOption("repos")
\end{verbatim}
The default, `factory fresh' setting will usually prompt you to select
a CRAN mirror.  One alternative for installing the package \WhatIf\ is
to use Gary King's website as the repository.  You can do this by
typing:
\begin{verbatim}
  > install.packages("WhatIf", repos = "http://gking.harvard.edu")
\end{verbatim}
You will then need to install the package \texttt{lpSolve} from CRAN by
typing:
\begin{verbatim}
  > install.packages("lpSolve")
\end{verbatim}
A second alternative is to download the Windows bundle from
\url{http://GKing.Harvard.Edu/bin/windows/contrib} and use the
\texttt{R} pull-down menu commands for installing a package from a zip
file.  You again will then need to install \texttt{lpSolve}, which can be
done either by typing the command given above at the command prompt or by
using the pull-down menus.  Finally, you then only need to type:
\begin{verbatim}
  > library("WhatIf") 
\end{verbatim}
from within \texttt{R} to load the \WhatIf\ package, after which you
may begin working with it.

\subsection{Linux/Unix}

You initially need to create both local \texttt{R} and local
\texttt{R} library directories if they do not already exist.  At the
Linux/Unix command prompt in your home directory, do this by typing:
  \begin{verbatim}
  > mkdir ~/.R ~/.R/library
  \end{verbatim}
Then open the `\texttt{.Renviron}' file that resides in your home directory,
creating it if necessary, and add the line:
  \begin{verbatim}
  R_LIBS = "~/.R/library"
  \end{verbatim}
using your preferred text editor.  These steps only need to be
performed once.  To install the package \WhatIf\ as well as the
package that it depends upon, \texttt{lpSolve}, type:
\begin{verbatim}
  > install.packages("WhatIf", dependencies = TRUE)
\end{verbatim}
at the \texttt{R} command prompt.  This command installs the packages
from the CRAN respository set as part of your \texttt{R} options.  You
can see what its current value is by calling
\begin{verbatim}
  > getOption("repos")
\end{verbatim}
The default, `factory fresh' setting will usually prompt you to select
a CRAN mirror.  One alternative for installing the package \WhatIf\ is
to use Gary King's website as the repository.  You can do this by
typing:
\begin{verbatim}
  > install.packages("WhatIf", repos = "http://gking.harvard.edu")
\end{verbatim}
A second alternative is to download the Linux/Unix bundle
`\texttt{WhatIf\_XX.tar.gz}', available from
\url{http://gking.harvard.edu/R/CRAN/src/contrib/}, and place it in
your home directory.  Note that `\texttt{XX}' is the current version
number.  Then, at the Linux/Unix command line from your home
directory, type
\begin{verbatim}
  > R CMD INSTALL WhatIf_XX.tar.gz
\end{verbatim}
to install the package.  Finally, you then only need to type:
\begin{verbatim}
  > library("WhatIf") 
\end{verbatim}
from within \texttt{R} to load the \WhatIf\ package, after which you
may begin working with it.

\section{Examples}

\subsection{Counterfactuals about U.N. Peacekeeping} \label{sec:cfeval}

This section illustrates the workings of \WhatIf\ with the empirical
example in Section $2.4$ of \citet{KinZen06}, which evaluates
counterfactuals about the causal impact of U.N.\ peacekeeping
operations on peacebuilding success.

The factual data set has 124 observations (including two with missing
values) on ten covariates as well as on the key causal variable,
\texttt{untype4}, which is a dummy variable.  The counterfactual data
set is the observed covariate data set with \texttt{untype4} replaced
with $1 - $\texttt{untype4}.  We list-wise delete the two
counterfactuals that are not fully observed.  We then save the two
data sets, one factual and the other counterfactual, as text files in
our current working directory and name them `\texttt{peacef.txt}' and
`\texttt{peacecf.txt}', respectively.  The first five rows of
`\texttt{peacef.txt}' look like:
\begin{verbatim}
  decade wartype   logcost wardur factnum factnumsq    trnsfcap untype4 treaty
1      5       1 14.917450     72       4        16    5.735545       0      0
2      4       0 15.671810    168       6        36    9.730863       0      0
3      5       1  6.907755     24       2         4   12.626030       0      0
4      5       1 12.971540     24       2         4 -112.000000       0      1
5      3       1  9.210340    216       2         4    4.275317       0      0
    develop       exp
1  132.8466 0.1217277
2  132.0000 0.1163292
3 1533.0000 0.0610000
4 2216.6080 0.1294513
5 1295.0000 0.1420000
\end{verbatim}
Similarly, the first five rows of `\texttt{peacecf.txt}' look like:
\begin{verbatim}
  decade wartype   logcost wardur factnum factnumsq    trnsfcap 1-untype4
1      5       1 14.917450     72       4        16    5.735545         1
2      4       0 15.671810    168       6        36    9.730863         1
3      5       1  6.907755     24       2         4   12.626030         1
4      5       1 12.971540     24       2         4 -112.000000         1
5      3       1  9.210340    216       2         4    4.275317         1
  treaty   develop       exp
1      0  132.8466 0.1217277
2      0  132.0000 0.1163292
3      0 1533.0000 0.0610000
4      1 2216.6080 0.1294513
5      0 1295.0000 0.1420000
\end{verbatim}

The function \texttt{whatif} can be called in two alternative ways
to analyze these counterfactuals.  First, typing:
  \begin{verbatim}
  > my.result <- whatif(data = "peacef.txt", cfact = "peacecf.txt") 
  \end{verbatim}
tells \texttt{whatif} to load the datasets `\texttt{peacef.txt}' and
`\texttt{peacecf.txt}' from our working directory.
Second, typing:
  \begin{verbatim}
  > my.result <- whatif(data = peacef, cfact = peacecf) 
  \end{verbatim}
tells \texttt{whatif} to use the \texttt{R} objects \texttt{peacef}
and \texttt{peacecf} loaded into memory prior to the function call.
These objects must be either non-character matrices or data frames
containing the counterfactual and observed covariate data,
respectively; in this case, they are data frames.  Alternatively, \texttt{peacef}
may be either a \hlink{Zelig}{/zelig} or other \texttt{R} model output
object (e.g., a model output object returned by a call to
\texttt{glm}).

The resulting output object \texttt{my.result} is a five-element list
(six-element if the option ``return.distance=T'' is used), each element of
which we now describe.  The first is simply the call.  The second is a
logical vector named \texttt{in.hull}, which contains the results of
the convex hull test.  Each element can have a value of either
\texttt{FALSE}, indicating that the corresponding counterfactual is
not in the convex hull of the observed data and thus requires
extrapolation, or \texttt{TRUE}, indicating the opposite.  To see the
values of \texttt{in.hull}, we type:
  \begin{verbatim}
  > my.result$in.hull 
  \end{verbatim} %$
For this example, the values are all \texttt{FALSE}.

The third element of the output list, \texttt{geom.var}, is the geometric
variability of the observed data, which we retrieve by typing:
  \begin{verbatim}
  > my.result$geom.var 
  \end{verbatim} %$
In this case, it is $0.110$ when rounding to three significant digits.
King and Zeng offer the geometric variability as a rule of thumb
threshold: counterfactuals with distances to the observed covariate
data less than this value are to some extent nearby the data.  By
default, pairwise Gower's distances ($G^{2}$) between each
counterfactual and data point are calculated by \texttt{whatif} in order
to determine which counterfactuals are nearby the data; alternatively, 
\texttt{whatif} will calculate the pairwise (squared) Euclidian
distance between each counterfactual and data point by setting the
parameter \texttt{distance} equal to \texttt{"euclidian"} as follows:
  \begin{verbatim}
  > my.result <- whatif(data = peacef, cfact = peacecf, distance = "euclidian")
  \end{verbatim}
However, this option is only appropriate for quantitative data; since
some of our variables are qualitative, we use the default Gower's
distance measure.  

Note that the matrix containing these distances can be large in size
and is not returned by default. To return the distance matrix, set the
parameter \texttt{return.distance} to \texttt{TRUE}.

The fourth element of the output object, \texttt{sum.stat}, is a numeric
vector, each element of which is the proportion of data points
nearby the corresponding counterfactual.  The values can be seen by
typing:
  \begin{verbatim}
  > my.result$sum.stat
  \end{verbatim} %$
The output looks like:
  \begin{verbatim}
            1           2           3           4           5           6 
  0.008196721 0.008196721 0.008196721 0.008196721 0.008196721 0.008196721 
            7           8           9          10          11          12
  0.008196721 0.008196721 0.008196721 0.008196721 0.008196721 0.008196721 
  ...
          121         122
  0.008196721 0.016393443
  \end{verbatim}
The numerical summary reported on page 14 of King and Zeng (2006) is
the average of \texttt{sum.stat} over all counterfactuals, which we can obtain
using the command
  \begin{verbatim}
  > mean(my.result$sum.stat)  
  \end{verbatim} %$
In this case, the average is $1.3$ percent.  This statistic is reported
for your convenience by the function \texttt{summary}.

We note that by default, `nearby' is defined as having a distance to
the counterfactual less than or equal to the geometric variability of
the observed data.  The default can be changed by setting a value for
the parameter \texttt{nearby}.  For example, to instead set the nearby
criterion at two geometric variances, we would type:
  \begin{verbatim}
  > my.result <- whatif(data = peacef, cfact = peacecf, nearby = 2)
  \end{verbatim}

The fifth element of the output object, \texttt{cum.freq}, stores information
on the cumulative frequency distribution of the distances
between a counterfactual and the observed covariate data.  To access
the cumulative frequency distribution for the default set of Gower
distances (from $0$ to $1$ in increments of $0.5$) between the first
counterfactual and the data points, for example, we type:
  \begin{verbatim}
  > my.result$cum.freq[1, ]
  \end{verbatim} %$
This prints the distribution to the screen:
  \begin{verbatim}
            0        0.05         0.1        0.15         0.2        0.25         
  0.000000000 0.000000000 0.008196721 0.081967213 0.262295082 0.483606557          
          0.3        0.35         0.4        0.45         0.5        0.55
  0.680327869 0.844262295 0.950819672 0.991803279 0.991803279 1.000000000
          0.6        0.65         0.7        0.75         0.8        0.85         
  1.000000000 1.000000000 1.000000000 1.000000000 1.000000000 1.000000000
          0.9        0.95           1
  1.000000000 1.000000000
  \end{verbatim}
Alternatively, we can change the default set of Gower distances by
using the parameter \texttt{freq}.  For example, to calculate a
cumulative frequency distribution solely for the Gower distances of $0$,
$0.5$, and $1.0$, we type:
  \begin{verbatim}
  > my.result <- whatif(data = peacef, cfact = peacecf, freq = c(0, 0.5, 1.0))
  \end{verbatim}
Now the cumulative frequency distribution for the first counterfactual
looks as follows:
  \begin{verbatim}
  > my.result$cum.freq[1, ]
          0       0.5         1 
  0.0000000 0.9918033 1.0000000 
  \end{verbatim} %$

We now turn to the auxiliary functions included in the \WhatIf\
package.  The first is \texttt{plot}, which produces figures that
graph the cumulative frequency distribution of the distances similar
to Figure 3 in King and Zeng (2006).  This function takes as its
input a \texttt{whatif} output object.  To plot the default cumulative
frequency distributions for all counterfactuals to the screen, type:
  \begin{verbatim}
  > plot(my.result)
  \end{verbatim}
Plotting 122 distributions on the same graph will not be very helpful,
however.  A particular frequency distribution or combination of
frequency distributions can be plotted by setting the parameter
\texttt{numcf} to equal the desired values.  For example, to plot
only the cumulative frequency distribution for the first
counterfactual, we type:
  \begin{verbatim}
  > plot(my.result, numcf = 1)
  \end{verbatim}
We also have the option of smoothing the raw cumulative
frequencies, which can be plotted either on their own or in
addition to the raw data.  The parameter controlling this option is
\texttt{type}.  To plot both the raw and LOWESS smoothed cumulative
frequency distributions for the first two counterfactuals, for
example, we type:
  \begin{verbatim}
  > plot(my.result, numcf = c(1, 2), type = "b")
  \end{verbatim}
where \texttt{"b"} stands for `both'.  Alternatively, assigning the value
\texttt{"l"} to \texttt{type} would plot only the smoothed frequencies.  To
save the graph as an encapsulated postscript file for later use 
instead of printing it to the screen, we set the parameter \texttt{eps}
equal to \texttt{TRUE}:
  \begin{verbatim}
  > plot(my.result, numcf = c(1, 2), type = "b", eps = TRUE)
  \end{verbatim}
The graph is saved to our working directory.

Not surprisingly, the function \texttt{summary} summarizes the most
important information produced by the function \texttt{whatif}.  The
output object, a list, contains this information, which may also be
printed to the screen.  For example, typing:
  \begin{verbatim}
  > summary(my.result)
  \end{verbatim}
displays the total number of counterfactuals evaluated; the number of
counterfactuals that are in the convex hull of the observed covariate
data; the percentage of data points nearby each counterfactual
averaged over all counterfactuals; and a table that contains both the
results of the convex hull test and the percentage of data points
nearby the counterfactual for each counterfactual.  Alternatively,
typing:
  \begin{verbatim}
  > my.result.sum <- summary(my.result)
  \end{verbatim}
saves the summary information as the object \texttt{my.result.sum},
which can be printed to the screen by typing either:
  \begin{verbatim}
  > print(my.result.sum)
  \end{verbatim}
or:
  \begin{verbatim}
  > my.result.sum
  \end{verbatim}
at the command prompt.

Finally, the package \WhatIf\ includes two print functions.  To print
the output object returned by \texttt{whatif} to the screen, type
either:
  \begin{verbatim}
  > print(my.result)
  \end{verbatim}
or the name of the output object at the command prompt.  Not printed
by these calls are the matrices of distances and cumulative
frequencies.  These large objects can be printed by setting the
parameters \texttt{print.dist} (if the distance matrix was returned)
and \texttt{print.freq} equal to \texttt{TRUE}, respectively.  For
example, to print the entire output object except for the matrix of
Gower distances to the screen, we type:
  \begin{verbatim}
  > print(my.result, print.freq = TRUE)
  \end{verbatim}
The other print function controls the printing of the output object
from the function \texttt{summary}.

\subsection{Identifying Common Support in Causal Inference}

The same algorithm for identifying whether or not counterfactuals
fall within the convex hull of the observed covariate data can be used
to assess common support.  We illustrate here with two examples.

\subsubsection{U.N. Peacekeeping}

In Section 2.4 of \citet{KinZen07}, the seven fully observed countries
that experienced a U.N.\ peacekeeping mission comprise the treatment
group while the remaining 115 fully observed (and 117 non-fully
observed) countries that lacked a U.N.\ peacekeeping mission
constitute the control group.  Cases of the former type receive a
coding of `1' on the key causal variable, \texttt{untype4}, while
cases of the latter type are coded `0'.

When estimating the \emph{average treatment effect on the treated}, we
discard controls with observed covariate data not within the convex
hull of the data for the treated as follows:
  \begin{verbatim}
  > my.result.cntrl <- whatif(formula = ~ decade + wartype + logcost +  
  + wardur + factnum + factnumsq + trnsfcap + treaty + develop + exp, 
  + data = peacef[peacef$untype4 == 1,], cfact = peacef[peacef$untype4 == 0,])
  \end{verbatim}
This command feeds the seven treated countries to the parameter
\texttt{data} and the 117 control countries to the parameter
\texttt{cfact}.  This differs from the last section, where
\texttt{data} contained all 124 observed data points, whether treated
or not.  The parameter \texttt{formula} allows us to drop \texttt{untype4}
from the two data frames, which we do by naming all of the variables
that we want to keep.  (We eliminate this variable since our goal is
to identify the convex hull of the observed pre-treatment covariates.)
Note that we do not specify a dependent variable in the formula.  We
then look at the results:
  \begin{verbatim}
  > my.result.cntrl$in.hull
  \end{verbatim} %$
The control group countries not on the support of the treated
countries are those with \texttt{FALSE} entries---in this case, all
115.  Note that for this data and call, two messages designed to
inform the user about choices made by \texttt{whatif} in the face of
problematic data are printed to the screen.  The first informs us that
\texttt{whatif} has deleted the two control group cases with missing
values from the \texttt{cfact} data set since counterfactuals must be
fully observed.  The second, ``range of at least one variable equals
zero'', informs us that \texttt{data} contains a degenerate case: the
variable \texttt{treaty} has no variance (and hence a range of zero)
in the observed covariate data set of the treated countries.  In order
to calculate the Gower distances, \texttt{whatif} must make
assumptions about the handling of such variables.  Specifically, it
ignores their contribution unless the values of the data point and
counterfactual are identical, in which case the normalized difference
is set to zero.

A different, but perhaps more reliably estimable quantity, may often
be obtained by \emph{also} dropping observations in the treatment
group whose observed covariate data falls outside of the convex hull
of the control group.  Any countries remaining comprise the data set
that lies on the common support.  Both the prior and this second step
can be performed simultaneously using \WhatIf\ as originally described
in Section \ref{sec:cfeval} for evaluating counterfactuals.
Accordingly, we type:
  \begin{verbatim}
  > peacef2cf <- peacef
  > peacef2cf$untype4 <- 1 - peacef2cf$untype4
  > my.result.comb <- whatif(data = peacef, cfact = peacef2cf)
  \end{verbatim}
Here, we initially create the counterfactual data set
\texttt{peacef2cf} directly from the factual, replacing
\texttt{untype4} with 1 - \texttt{untype4}.  We could also have
supplied the data set \texttt{peacecf}, originally constructed in a
similar manner.  (The data sets would be identical if the two
counterfactuals with missing data were list-wise deleted from
\texttt{peacef2cf}.)  We now look at the results of the convex hull
test as before and see that none of the counterfactuals are in the
convex hull.  Hence, there is no data on the common support of both
the treatment and control groups.

\subsubsection{Hypothetical Data}

To demonstrate that the latter approach really does combine the
individual assessments of support on the treatment and control groups,
consider this hypothetical data set:
  \begin{verbatim}
  > sqdata <- data.frame(t = c(1, 1, 1, 1, 0, 0, 0, 0), 
  +                      x = c(0, 0, 1, 1, .5, .5, 1.5, 1.5), 
  +                      y = c(1, 0, 0, 1, .5, 1.5, .5, 1.5))
  \end{verbatim}
The variable `t' is the treatment.  The convex hull of the observed
covariate data of the treatment group is obviously a unit square with
its lower left vertex at the origin.  The convex hull of the control
group is also a unit square, but one with its lower left vertex at the
point $(0.5, 0.5)$ in the Cartesian plane.

We first identify the control group units that are not on the support
of the treated units (i.e., the control group units not in the convex
hull of the covariate data of the treated group) as follows:
  \begin{verbatim}
  > summary(whatif(~ x + y, data = sqdata[sqdata$t == 1,], cfact = 
  + sqdata[sqdata$t == 0,]))
  \end{verbatim}
which as before uses the parameter \texttt{formula} to eliminate the
treatment variable, \texttt{t}, from the data frames.  Only the first unit
from the control group, the point $(0.5, 0.5)$ in the Cartesian plane,
is in the convex hull and hence on the support of the treated group.
We next identify the treated group units that are not on the support
of the control units by typing:
  \begin{verbatim}
  > summary(whatif(~ x + y, data = sqdata[sqdata$t == 0,], cfact = 
  + sqdata[sqdata$t == 1,]))
  \end{verbatim}
The treatment group unit represented by the point $(1, 1)$ is the only
one in the convex hull and hence on the support of the control group.
Accordingly, if we were to eliminate the units without common support
as identified by the two separate tests, we would eliminate all units
save the points $(0.5, 0.5)$, the only control group unit on the support
of the treated group, and $(1, 1)$, the only treated group unit on the
support of the control group.

Alternatively, we can combine the two steps:
  \begin{verbatim}
  > summary(whatif(data = sqdata, cfact = cbind(1 - sqdata[, 1], sqdata[, 2:3])))
  \end{verbatim}
This time, two counterfactuals are in the convex hull of the data.
These counterfactuals correspond to the units with observed covariate
data $(0.5, 0.5)$ and $(1, 1)$.  Accordingly, we conclude that only two
units are on the common support, the same conclusion that we drew from
the two separate tests.

\subsection{Using WhatIf with Zelig}

We now illustrate how \WhatIf\ can be easily used with \texttt{Zelig}.
As an example, we first generate \texttt{Zelig} output from a simple
logistic model using the hypothetical data set created in the prior
example:
  \begin{verbatim}
  > z.out <- zelig(t ~ x + y, data = sqdata, model = "logit")
  \end{verbatim}
We next create a counterfactual using the \texttt{Zelig} command
\texttt{setx}:
  \begin{verbatim}
  > x.out <- setx(z.out, x = 2, y = 3)
  \end{verbatim}
This is normally followed by a call to the \texttt{Zelig} command
\texttt{sim} to compute quantities of interest, such as predicted
values given these values of the explanatory variables.  See, for
example,
\hlink{http://gking.harvard.edu/zelig/docs/Quick\_Overview.html}
{/zelig/docs/Quick_Overview.html}.  \WhatIf\ enables you to evaluate
the values to which you set the explanatory variables before
simulating quantities of interest.  We do this by calling
\texttt{whatif} as follows:
  \begin{verbatim}
  > summary(whatif(data = z.out, cfact = x.out))
  \end{verbatim}
The results indicate that this counterfactual is not in the convex
hull of the data.  In this situation, you may want to rethink whether
or not you should proceed on to the \texttt{sim} stage of analysis.
Note that if an intercept was fit as part of the original model,
\texttt{whatif} automatically drops it from both the observed
covariate data set extracted from the \texttt{zelig} output object
\texttt{z.out} and the \texttt{setx}-generated counterfactual \texttt{x.out}.

\subsection{Using WhatIf with Other \texttt{R} Model Output Objects}

Suppose that instead of using \texttt{Zelig}, we use the function
\texttt{lm} to fit a linear model to the same hypothetical data
by typing:
\begin{verbatim}
  > lm.out <- lm(t ~ x + y, data = sqdata)
\end{verbatim}
In this case, we could then use \WhatIf\ to evaluate a counterfactual
as follows:
\begin{verbatim}
  > summary(whatif(data = lm.out, cfact = data.frame(x = 2, y = 3)))
\end{verbatim}
As with \texttt{zelig} output objects, intercepts are dropped from the
observed covariate data sets extracted in this manner.  Unlike with
\texttt{Zelig}, however, counterfactuals are not generated
automatically by \texttt{lm}; hence, the counterfactuals that you
supply to \texttt{whatif} should not include an intercept.  The
parameter \texttt{formula} can be used to drop, select, and transform
the variables in \texttt{data} and \texttt{cfact} when \texttt{data}
is a \texttt{R} model or \texttt{zelig} output object in the same way
that it can be used when \texttt{data} is a matrix or data frame.  For
example, to drop the variable \texttt{x}, we type:
\begin{verbatim}
  > summary(whatif(~ y, data = lm.out, cfact = data.frame(x = 2, y = 3)))
\end{verbatim}
or more simply and equivalently:
\begin{verbatim}
  > summary(whatif(~ y, data = lm.out, cfact = data.frame(y = 3)))
\end{verbatim}
If instead we decide to run the test using the square of \texttt{x}, we type:
\begin{verbatim}
  > summary(whatif(~ I(x^2) + y, data = lm.out, cfact = data.frame(x = 2, y = 3)))
\end{verbatim}
following standard \texttt{R} conventions for formulas.

\subsection{Demos and Data Sets}

\texttt{R} will automatically walk you through the examples related to
U.N.\ peacekeeping by running
  \begin{verbatim}
  > demo("peace")
  \end{verbatim}
The factual and counterfactual U.N.\ peacekeeping data sets used in
the examples are included in the \WhatIf\ package.  You may load them
by calling:
  \begin{verbatim}
  > data("peacef")
  \end{verbatim}
and
  \begin{verbatim}
  > data("peacecf")
  \end{verbatim}
which stores them as \texttt{R} objects with the corresponding names.

\section{Technical Details}

The computational task of determining the convex hull membership is
made feasible even for large numbers of explanatory variables and
observations by the solution proposed in \citet{KinZen06}, which
eliminates the most time-consuming part of the problem: the
characterization of the convex hull itself.  In addition, they show
that the remaining (implicit) point location problem can be expressed
as a linear programming exercise, making it possible to take advantage
of existing well-developed algorithms designed for other purposes to
speed up the computation.  Specifically, a counterfactual $x$ is in
the convex hull of the explanatory variables $X$ if there exists a
feasible solution to the following standard form linear programming
problem:
\begin{align}
\label{lp}
\text{min }  & C'\eta           \nonumber\\
\text{s.t. } & A'\eta=B'                \\
            & \eta \geq 0      \nonumber
\end{align}
%\begin{equation}
%\mbox{min } C' \eta
%\end{equation}
%\begin{equation}
%\mbox{s. t. } A' \eta = B' \mbox{ and } \eta \geq 0
%\end{equation}
where $C$ is a vector of zeros (so that there is no objective function
to minimize); $\eta$ is a vector of coefficients; $A'$ is $X'$ with an
additional, final row of $1$'s; and $B'$ is $x'$ with an additional,
final element equal to $1$.

The default Gower distance (which is suitable for both quantitative
and qualitative data) between a pair of $K$ dimensional points $x_i$
and $x_j$ is defined simply as the average absolute distance between
the elements of the two points divided by the range of the data:
\begin{equation}
  \label{gower}
  G_{ij} = \frac{1}{K}\sum_{k=1}^K \frac{\left| x_{ik} - 
x_{jk}\right|}{r_k}
\end{equation}
where the range is $r_k = \text{max}(X_{.k}) - \text{min}(X_{.k})$ and
the min and max functions return the smallest and largest elements
respectively in the set including the $k$th element of the explanatory
variables $X$.  The optional squared Euclidian distance (which is
suitable only for quantitative data) between points $x_{i}$ and
$x_{j}$ is given by the familiar definition, i.e. the sum of the
squared differences between the elements of the two points:
\begin{equation}
  E_{ij} = \sum_{k=1}^K (x_{ik} - x_{jk})^2 \mbox{ .}
\end{equation}

\section{\texttt{R} Function Reference}

\subsection{Function \texttt{whatif()}}

This function evaluates your counterfactuals.  Specifically, it:
\begin{enumerate}
\item Determines if your counterfactuals are in the convex hull of the
  observed covariate data and are therefore interpolations or if they instead
  lie outside of it and are therefore extrapolations.
\item Computes the distance from your counterfactuals to
  each of the $n$ observed data points.  The default distance function used is
  Gower's non-parametric measure.  \label{item:dist}
\item Computes a summary statistic for each counterfactual based on
  the distances in \ref{item:dist}: the fraction of observed covariate data points with
  distances to your counterfactual less than a value you supply.
  By default, this value is taken to be the geometric variability of the
  observed data.
\item Computes the cumulative frequency distribution of each
  counterfactual for the distances in \ref{item:dist} using values that you
  supply.  By default, Gower distances from $0$ to $1$ in increments of
  $0.05$ are used.
\end{enumerate}
In other words, this function provides you with both qualitative and
quantitative information about your counterfactuals, including two
numeric summaries.  You can then feed the output of this function
either to \texttt{plot} to generate a graphical view or to
\texttt{summary} to get a numerical summary of the results.

\subsubsection{Usage}

\begin{verbatim}
     whatif(formula = NULL, data, cfact, range = NULL, freq = NULL, 
     nearby = 1, distance = "gower", miss = "list", choice = "both",
     return.inputs = FALSE, return.distance = FALSE, ...)
\end{verbatim}

\subsubsection{Inputs}

\begin{description}
\item[formula] An optional formula without a dependent variable that
  is of class ``formula'' and that follows standard \texttt{R}
  conventions for formulas, e.g. \verb1~1 x1 + x2.  Allows you to
  transform or otherwise re-specify combinations of the variables in
  both \texttt{data} and \texttt{cfact}.  To use this parameter, both
  \texttt{data} and \texttt{cfact} must be coercable to data frames;
  the variables of both \texttt{data} and \texttt{cfact} must be
  labeled; and all variables appearing in \texttt{formula} must also
  appear in both \texttt{data} and \texttt{cfact}.  Otherwise, errors
  are returned.  The intercept is automatically dropped.  Default is
  \texttt{NULL}.
\item[data] May take one of the following forms:
  \begin{enumerate}
  \item A \texttt{R} model output object, such as the output from calls to
    \texttt{lm}, \texttt{glm}, and \texttt{zelig}.  Such an output
    object must be a list.  It must additionally have either a \texttt{formula}
    or \texttt{terms} component and either a \texttt{data} or \texttt{model} component; if it
    does not, an error is returned.  Of the latter, \texttt{whatif}
    first looks for \texttt{data}, which should contain either the original
    data set supplied as part of the model call (as in \texttt{glm})
    or the name of this data set (as in \texttt{zelig}), which is
    assumed to reside in the global environment.  If \texttt{data} does not
    exist, \texttt{whatif} then looks for \texttt{model}, which should
    contain the model frame (as in \texttt{lm}).  The intercept is
    automatically dropped from the extracted \emph{observed covariate
    data} set if the original model included one.
  \item A $n \times k$ non-character (logical or numeric) matrix or data
    frame of \emph{observed covariate data} with $n$ data points or
    units and $k$ covariates.  All desired variable transformations
    and interaction terms should be included in this set of $k$
    covariates unless \texttt{formula} is alternatively used to
    produce them.  However, an intercept should not be.  Such a matrix
    may be obtained by passing model output (e.g., output from a call
    to \texttt{lm}) to \texttt{model.matrix} and excluding the 
    intercept from the resulting matrix if one was fit.  Note
    that \texttt{whatif} will attempt to coerce data frames to their
    internal numeric values.  Hence, data frames should only contain
    logical, numeric, and factor columns; character columns will lead
    to an error being returned.
  \item A string.  Either the complete path (including file name) of
    the file containing the data or the path relative to your working
    directory.  This file should be a white space delimited text file.
    If it contains a header, you must include a column of row names as
    discussed in the help file for the \texttt{R} function
    \texttt{read.table}.  The data in the file should be as otherwise
    described in (2).
  \end{enumerate}
  Missing data is allowed and will be dealt with via the argument
  \texttt{missing}.  It should be flagged using \texttt{R}'s
  standard representation for missing data, \texttt{NA}.
\item[cfact] A \texttt{R} object or a string.  If a \texttt{R} object,
  a $m \times k$ non-character matrix or data frame of
  \emph{counterfactuals} with $m$ counterfactuals and the same $k$
  covariates (in the same order) as in \texttt{data}.  However, if
  \texttt{formula} is used to select a subset of the $k$ covariates,
  then \texttt{cfact} may contain either only these $j \leq k$
  covariates or the complete set of $k$ covariates.  An intercept
  should not be included as one of the covariates.  It will be
  automatically dropped from the counterfactuals generated by
  \texttt{Zelig} if the original model contained one.  Data frames
  will again be coerced to their internal numeric values if possible.
  If a string, either the complete path (including file name) of the
  file containing the counterfactuals or the path relative to your
  working directory.  This file should be a white space delimited text
  file.  See the discussion under \texttt{data} for instructions on
  dealing with a header.  All counterfactuals should be fully
  observed: if you supply counterfactuals with missing data, they will
  be list-wise deleted and a warning message will be printed to the
  screen.
\item[range] An optional numeric vector of length $k$, where $k$ is
  the number of covariates.  Each element represents the range of the
  corresponding covariate for use in calculating Gower distances.  Use
  this argument when covariate data do not represent the population of
  interest, such as selection by stratification or experimental
  manipulation.  By default, the range of each covariate is calculated
  from the data (the difference of its maximum and minimum values in
  the sample), which is appropriate when a simple random sampling
  design was used.  To supply your own range for the $k$th covariate,
  set the $k$th element of the vector equal to the desired range and
  all other elements equal to \texttt{NA}.  Default is \texttt{NULL}.
\item[freq] An optional numeric vector of any positive length, the
  elements of which comprise a set of distances.  Used in calculating
  cumulative frequency distributions for the distances of the data
  points from each counterfactual.  For each such distance and
  counterfactual, the cumulative frequency is the fraction of observed
  covariate data points with distance to the counterfactual less than
  or equal to the supplied distance value.  The default varies with
  the distance measure used.  When the Gower distance measure is
  employed, frequencies are calculated for the sequence of Gower
  distances from $0$ to $1$ in increments of $0.05$.  When the Euclidian
  distance measure is employed, frequencies are calculated for the
  sequence of Euclidian distances from the minimum to the maximum
  observed distances in twenty equal increments, all rounded to two
  decimal places. Default is \texttt{NULL}.
\item[nearby] An optional scalar indicating which observed data
  points are considered to be nearby (i.e., within `nearby' geometric
  variances of) the counterfactuals.  Used to calculate the summary
  statistic returned by the function: the fraction of the observed
  data nearby each counterfactual.  By default, the geometric
  variability of the covariate data is used.  For example, setting
  \texttt{nearby} to $2$ will identify the proportion of data points
  within two geometric variances of a counterfactual.  Default is \texttt{1}.
\item[distance] An optional string indicating the distance measure to
  employ.  The choices are either \texttt{"gower"}, Gower's
  non-parametric distance measure ($G^{2}$), which is suitable for
  both qualitative and quantitative data; or \texttt{"euclidian"},
  (squared) Euclidian distance, which is only suitable for quantitative
  data.  The default is the former, \texttt{"gower"}.
\item[miss] An optional string indicating the strategy for dealing
  with missing data in the observed covariate data set.
  \texttt{whatif} supports two possible missing data strategies:
  \texttt{"list"}, list-wise deletion of missing cases; and
  \texttt{"case"}, ignoring missing data case-by-case (pairwise
  deletion).  Note that if \texttt{"case"} is selected, observations
  with missing values are still deleted listwise for the convex hull
  test and for computing Euclidian distances, but pairwise deletion is
  used in computing the Gower distances to maximally use available
  information. The user is strongly encouraged to treat missing data
  using specialized tools such as Amelia prior to feeding the data to
  \texttt{whatif}.  Default is \texttt{"list"}.
\item[choice] An optional string indicating which analyses to 
  undertake. The options are either \texttt{"hull"}, only perform the convex hull 
  membership test; \texttt{"distance"}, do not perform the convex
  hull test but do everything else, such as calculating the distance between
  each counterfactual and data point; or \texttt{"both"}, undertake both the
  convex hull test and the distance calculations (i.e., do everything).
  Default is \texttt{"both"}.
\item[return.inputs] A Boolean; should the processed observed
  covariate and counterfactual data matrices on which all
  \texttt{whatif} computations are performed be returned?  Processing
  refers to internal \texttt{whatif} operations such as the subsetting
  of covariates via \texttt{formula}, the deletion of cases with
  missing values, and the coercion of data frames to numeric matrices.
  Primarily intended for diagnostic purposes.  If \texttt{TRUE}, these matrices
  are returned as a list.  Default is \texttt{FALSE}.
\item[return.distance] A Boolean; should the matrix of distances
  between each counterfactual and data point be returned?  If
  \texttt{TRUE}, this matrix is returned as part of the output; if
  \texttt{FALSE}, it is not.  Default is \texttt{FALSE} due to the large
  size that this matrix may attain.
\end{description}

\subsubsection{Value}

An object of class ``whatif'', a list containing the following six or
seven elements:
\begin{description}
\item[call]  The original call to \texttt{whatif}.
\item[inputs] A list with two elements, \texttt{data} and \texttt{cfact}.  Only
  present if \texttt{return.inputs} was set equal to \texttt{TRUE} in the call
  to \texttt{whatif}.  The first element is the processed observed
  covariate data matrix on which all \texttt{whatif} computations were
  performed.  The second element is the processed counterfactual data
  matrix.
\item[in.hull] A logical vector of length $m$, where $m$ is
  the number of counterfactuals.  Each element of the vector is \texttt{TRUE}
  if the corresponding counterfactual is in the convex hull and \texttt{FALSE}
  otherwise.
\item[dist] An $m \times n$ numeric matrix, where $m$ is the number of
  counterfactuals and $n$ is the number of data points (units).  Only
  present if \texttt{return.distance} was set equal to \texttt{TRUE}
  in the call to \texttt{whatif}.  The $[i,j]$th entry of the matrix
  contains the distance between the $i$th counterfactual and the $j$th
  data point.
\item[geom.var]  A scalar.  The geometric variability of the observed
  covariate data.
\item[sum.stat] A numeric vector of length $m$, where $m$ is the
  number of counterfactuals.  The $m$th element contains the summary
  statistic for the corresponding counterfactual.  This summary
  statistic is the fraction of data points with distances to the
  counterfactual less than \texttt{nearby*gv}, which by
  default is the geometric variability of the covariates.
\item[cum.freq] A numeric matrix.  By default, the matrix has
  dimension $m \times 21$, where $m$ is the number of counterfactuals;
  however, if you supplied your own frequencies via the argument
  \texttt{freq}, the matrix has dimension $m \times f$, where $f$ is the
  length of \texttt{freq}.  Each row of the matrix contains the
  cumulative frequency distribution for the corresponding
  counterfactual calculated using either the distance measure-specific
  default set of distance values or the set that you supplied (see the
  discussion under the argument \texttt{freq}).  Hence, the $[i,j]$th
  entry of the matrix is the fraction of data points with 
  distances to the $i$th counterfactual less than or equal to the
  value represented by the $j$th column.  The column names contain these
  values.
\end{description}

\subsection{Function \texttt{plot.whatif()}}

This function generates a cumulative frequency plot of distances,
graphically summarizing the distance of your counterfactuals from the
data.  It takes as its input an object returned by the function
\texttt{whatif}, which has class ``whatif''.  It is called by the
generic function \texttt{plot}.  The cumulative frequencies appear on
the vertical axis.  They are the fraction of rows in the observed data
set with either Gower or (squared) Euclidian distances to the
counterfactuals less than or equal to the given value on the
horizontal axis.  Counterfactuals in the convex hull are plotted with
a solid line and counterfactuals outside of the convex hull with a
dashed line.

\subsubsection{Usage}

\begin{verbatim}
     plot(x, type = "f", numcf = NULL, eps = FALSE, ...)
\end{verbatim}

\subsubsection{Inputs}

\begin{description}
\item[x] An object of class ``whatif'', the output of the function
  \texttt{whatif()}.
\item[type] A character string; the type of plot of the cumulative
  frequencies of the distances to be produced, including:
  \texttt{"f"} for cumulative frequencies only; \texttt{"l"} for
  LOWESS smoothing of cumulative frequencies only; \texttt{"b"} for
  both cumulative frequencies and LOWESS smoothing.  LOWESS
  scatterplot smoothing is plotted in blue and the unsmoothed
  frequencies in black.  Default is \texttt{"f"}.
\item[numcf] A numeric vector; the specific counterfactuals to be
  plotted.  Each element represents a counterfactual, specifically its
  row number from the matrix or data frame of counterfactuals.  By default, all
  counterfactuals are plotted.  Default is \texttt{NULL}.
\item[eps] A Boolean; should an encapsulated postscript file be
  generated?  Setting the argument equal to \texttt{TRUE} generates an
  \texttt{.eps} file, which is saved to your working directory with
  file name of form `\texttt{graph\_`type'\_`numcf'.eps}', where \texttt{`type'}
  and \texttt{`numcf'} are the values of the respective arguments.  Specifically, 
  \texttt{`numcf'} takes the value of the first element of the argument \texttt{numcf}  
  unless all counterfactuals were plotted, in which case \texttt{all} appears in the place of 
  \texttt{`numcf'}.  Default is \texttt{FALSE}, which instead 
  prints the graph to the screen.
\end{description}

\subsubsection{Value}

Either a graph printed to the screen or an encapsulated postscript
file saved to your working directory.  In the latter case, the file
name has form `\texttt{graph\_`type'\_`numcf'.eps}', where \texttt{`type'} and
\texttt{`numcf'} are the values of the respective arguments.

\subsection{Function \texttt{print.whatif()}}

This function prints the information generated by a call to
\texttt{whatif} to the screen.  It takes as its input an object
returned by the function \texttt{whatif}, which has class ``whatif''.
It is called by the generic function \texttt{print}.

\subsubsection{Usage}

\begin{verbatim}
     print(x, print.dist = FALSE, print.freq = FALSE, ...)
\end{verbatim}

\subsubsection{Inputs}

\begin{description}
  \item[x] An object of class ``whatif'', the output of the
    function \texttt{whatif}.
  \item[print.dist] A Boolean; should the matrix of pairwise 
    distances between each counterfactual and data point be printed to
    the screen, if it was returned?  Default is \texttt{FALSE}.
  \item[print.freq] A Boolean; should the matrix of cumulative
    frequencies of distances for each counterfactual be printed
    to the screen?  Default is \texttt{FALSE}.
\end{description}

\subsubsection{Value}

A printout to the screen of the information contained in the 
\texttt{whatif} output object.

\subsection{Function \texttt{print.summary.whatif()}}

This function prints the information generated from the
\texttt{whatif} output object by a call to \texttt{summary} to the
screen.  It takes as its input an object returned by the function
\texttt{summary.whatif}, which has class ``summary.whatif''.  It is
called by the generic function \texttt{print}.

\subsubsection{Usage}

\begin{verbatim}
     print(x, ...)
\end{verbatim}

\subsubsection{Inputs}

\begin{description}
  \item[x]  An object of class ``summary.whatif'', the output of
    the function \texttt{summary.whatif}.
\end{description}

\subsubsection{Value}

A printout to the screen of the \texttt{whatif} information summarized
in the \texttt{summary.whatif} output object.

\subsection{Function \texttt{summary.whatif()}}

This function summarizes the information produced by \texttt{whatif}
about your counterfactuals.  It takes as its input the object returned
by the function \texttt{whatif}, which has class ``whatif''.  It is
called by the generic function \texttt{summary}.  The summary generated
is returned as well as printed to the screen.

\subsubsection{Usage}

\begin{verbatim}
     summary(object, ...)
\end{verbatim}

\subsubsection{Inputs}

\begin{description}
  \item[object]  An object of class ``whatif'', the output of 
    the function \texttt{whatif}.
\end{description}

\subsubsection{Value}

An object of class ``summary.whatif'', a list containing the following five
elements:
\begin{description}
\item[call]  The original call to \texttt{whatif}.
\item[m] A scalar.  The total number of counterfactuals evaluated.
\item[m.inhull] A scalar.  The number of counterfactuals evaluated
  that are in the convex hull of the observed covariate data.
\item[mean.near] A scalar.  The average percentage of data `nearby'
  each counterfactual, where the average is taken over all
  counterfactuals.
\item[sum.df] A data frame with three columns and $m$ rows, where $m$ is
  the number of counterfactuals.  The first column, \texttt{cfact}, indexes
  the counterfactuals.  The second column, \texttt{in.hull}, contains the
  results of the convex hull test.  The third column, \texttt{per.near},
  contains the percentage of data points nearby each counterfactual.
\end{description}
This object is printed to the screen.

\bibliographystyle{apsr}
\bibsep=0in
\bibliography{gk.bib,gkpubs.bib}
\end{document}

